This document will mainly touch on basics of SV creation, most of the advanced work on SVs should be done by you.
This will not touch on anything advanced, such as:
\begin{itemize}
\item Complex SV creations
\item BPM Mapping
\item Automated Creation of TimingPoints
\item p-Notes / Fake Notes
\end{itemize}

I’ll try to be as unbiased as possible when writing this document, please take this as a point of reference rather than a rule of thumb, the limitations much further than what this document will cover.

\section{Starting Out}

\subsection{Adding your First SVs}
\begin{itemize}
    \item Ctrl+Shift+P adds an SV Line
    \item Ctrl+P adds a BPM Line
\end{itemize}
Note for \textbf{new mappers}: \newline 
\textit{It is possible to add multiple SV/BPM lines at one single time stamp. For most cases, SVs on top of SVs \textbf{OR} BPMs on top of BPMs will generate \underline{undefined behavior}, so try to avoid it. \newline
However, SVs on top of BPMs may be used to \textbf{Normalize}, in which will be a future covered topic.}

\subsection{Defining your SV Values}
The value directly \textbf{multiplies into} the scroll speed the player is currently running on, so \textbf{2.0} scrolls \underline{twice as fast} as \textbf{1.0}.
\subsubsection{Boundaries}
osu!mania only supports $0.1 \longleftrightarrow 10.0$. \newline Contrary to being able to use $0.01 \longleftrightarrow 0.10$, there is no effect on the map.

\subsection{BPM Effects}
If you have encountered a multi-BPM map, you'll notice that BPM lines will affect scroll speed too. It works similar to SVs, where \textbf{200 BPM} sections will scroll \underline{twice as fast} as \textbf{100 BPM} sections.

\subsection{Always consider Slider Velocities over BPM Lines}
There are 2 ways to manipulate scroll speed, through BPM and SV. You can take it as a rule of thumb to \textbf{NEVER} use BPM to change scroll speed unless you know what you're doing! I'll list down the pros and cons below. \newline
Take note that the context is to \underline{only use BPM for scroll speed manipulation}

\subsubsection{Pros of BPM usage}
\begin{itemize}
    \item Very wide limits $(0 \longleftrightarrow inf)$
    \item Can generate/affect measure lines
    \item Arguably easier to calculate when used in Multi-BPM maps
\end{itemize}

\subsubsection{Cons of BPM usage}
\begin{itemize}
    \item Not intuitive on creation of extreme values
    \item Inaccuracies when rounding to 3 decimal places \textit{$eg. \sim 33.333$}
    \item Due to osu!mania being reliant on snap mapping, it will be not intuitive to map with an altered BPM when creating effects.
    \item BPM Lines will lie on integer offsets, hence with non-integer beat lengths, the original BPM offset will start to sway.
    \item Extreme BPM lines can crash the editor
    \item And many more...
\end{itemize}
\underline{To summarize, do not use BPM Lines unless you know what you are doing.} \newline
\textbf{I will not be covering BPM scroll speed manipulation in this document.}