This will be the final chapter in the document. So if you've understood everything so far, you're well on your way to mastering it already.

\section{Visualizing Everything}

\subfile{../part_5/figs/ae_tlfig1.tex}

How would we get the value required to create \textbf{Even Visual Spacing}? The simple answer is to refer back to the diagrams we drew previously and work from there.

\subfile{../part_5/figs/ae_diagfig1.tex}
\newline
One way we can deduce the value there is through \textbf{calculation of Area}.
\newline
Before we delve into the calculation, we need to first know what is the shaded area called. I personally coin it as \textbf{Distance} because it defines how far the notes will travel, and \textbf{Distance} defined as:

\[ SV * Offset = Distance \]
Important Rule of Thumb: 
\begin{center}
   \textbf{SV Sequences with equal Distances will have Even Visual Spacing with respect to each other}
\end{center}
\newpage

\subsubsection{Calculation with Geometry}

This is the simpler but longer way of calculating the unknown. If you prefer using \textbf{equations} you can skip to the following section.

First and foremost, we need to understand that in both diagrams, they have the same shaded area (Distance), so if we can calculate the \textit{shifted area} (indicated by the arrow), we can calculate X.

We first calculate the shifted area as shown:

\[ Area_B = SV * Offset = 2.0 * 20 = 40 \]
\[ Area_B + Area_C = SV * Offset = 1.0 * 80 = 80 \]
\[ Area_C = (Area_B + Area_C) - Area_B = 80 - 40 = 40 \]
\[ Area_C = SV * Offset = X * 80 = 40 \]
\[ X = 0.5 \]

Note: There are some steps that you can avoid with different approaches.

\subsubsection{Calculation with Equations}

This is the harder but shorter way of calculating the unknown. If you prefer using \textbf{Geometry} read the previous section.

Looking at the \textbf{Left Diagram}, we can calculate \textbf{Distance}:

\[ Distance = SV_{Total} * Offset_{Total} = 1.0 * 100 = 100 \]

Note: We can usually skip this step and assume $Distance = Offset$ because most of the time we use 1.0x as the $SV_{Total}$, and we know anything multiplied by 1.0 is the same.\newline
We know that the total of the colored area (Distance) never changes because \underline{Distance remains the same} so that \underline{Visual Spacing remains even}. We can then calculate $X$.

\[ SV_1 * Offset_1 + SV_2 * Offset_2 = Distance = 100\]
\[ 3.0 * 20 + X * 80 = 100\]
\[ X = 0.5\]

These are the 2 ways you can calculate unknown SVs.
Let's take a look at more examples: \newpage

\subfile{../part_5/figs/ae_tlfig2.tex}

\subfile{../part_5/figs/ae_diagfig2.tex}

\subsubsection{Calculation with Geometry}

\[ Area_A = SV * Offset = 0.2 * 60 = 12 \]
\[ Area_A + Area_B = SV * Offset = 1.0 * 60 = 60 \]
\[ Area_B = (Area_A + Area_B) - Area_A = 60 - 12 = 48 \]
\[ Area_B = SV * Offset = (X - 1.0) * 40 = 48 \]
\[ X = 1.2 \]

\subsubsection{Calculation with Equations}

Calculation of Expected Distance:
\[ Distance = SV_{Total} * Offset_{Total} = 1.0 * 100 = 100 \]

Solving the Equation:
\[ SV_1 * Offset_1 + SV_2 * Offset_2 = Distance = 100\]
\[ 0.2 * 60 + X * 40 = 100\]
\[ X = 1.2\]
\newpage

\subfile{../part_5/figs/ae_tlfig3.tex}

\subfile{../part_5/figs/ae_diagfig3.tex}

\subsubsection{Calculation with Geometry}

\[ Area_B = SV * Offset = 1.25 * 20 = 25 \]
\[ Area_B + Area_C = SV * Offset = 0.75 * 80 = 60 \]
\[ Area_C = (Area_B + Area_C) - Area_B = 60 - 25 = 35 \]
\[ Area_C = SV * Offset = X * 80 = 35 \]
\[ X = 0.4375 \approx 0.44 (2d.p.) \]

\subsubsection{Calculation with Equations}

Note the $SV_{Total}$.

\textcolor{red}{Calculation of Expected Distance:}
\[ Distance = SV_{Total} * Offset_{Total} = 0.75 * 100 = 75 \]

Solving the Equation:
\[ SV_1 * Offset_1 + SV_2 * Offset_2 = Distance = 75\]
\[ 2.0 * 20 + X * 80 = 75\]
\[ X = 0.4375 \approx 0.44 (2d.p.)\]
\newpage

\subfile{../part_5/figs/ae_tlfig4.tex}

\subfile{../part_5/figs/ae_diagfig4.tex}

\subsubsection{Calculation with Geometry}

\[ Area_B = SV * Offset = 1.0 * 20 = 20 \]
\[ Area_B + Area_C = SV * Offset = 1 * 40 = 40 \]
\[ Area_C = (Area_B + Area_C) - Area_B = 40 - 20 = 20 \]
\[ Area_C = SV * Offset = X * 40 = 20 \]
\[ X = 0.5 \]

\subsubsection{Calculation with Equations}

Note the $Offset_{Total}$.

\textcolor{red}{Calculation of Expected Distance:}
\[ Distance = SV_{Total} * Offset_{Total} = 1.0 * 60 = 60 \]

Solving the Equation:
\[ SV_1 * Offset_1 + SV_2 * Offset_2 = Distance = 60\]
\[ 2.0 * 20 + X * 40 = 60\]
\[ X = 0.5\]
\newpage