\section{Act, Counteract and Closure}
For one of the simplest and most common effects, most mappers goes through 3 steps of creation
\begin{itemize}
	\item Act: The main effect you want to create
	\item Counteract: The effect that \textit{evens out the Visual Spacing}
	\item Closure: This is usually to normalize the SV back to 1.0, as to "close" the sequence
\end{itemize}
Let's take a look at each of these in detail and what they do.

\subsubsection{The Act}
As explained, this is the \textbf{main source} of the effect. What you want to achieve in the effect should be done here. \newline
Take for example:
\begin{itemize}
	\item \textbf{An effect that noticeably jumps forward once a note is hit} \newline
		  We should expect a \textit{reasonably high value SV} to achieve this effect.\newline For simplicity, we will only use up to $1.9$, I will explain why in the \textbf{Half-half Concept} topic.
	\item \textbf{An weak effect that will barely affect gameplay, but enough for players to notice once a note is hit} \newline
		  We should expect a \textit{relatively weak SV} as compared to the above example. \newline
		  We can consider anything in the range of $1.2 \longleftrightarrow 1.5$ where is tame, but not too close to 1.0.
\end{itemize}

\subsubsection{The Counteract}
As you'll expect, you will end up with \textbf{Uneven Visual Spacing} if you just stop here, so this one mainly targets that issue and corrects it. We'll go into detail how to do this later.

\subsubsection{The Closure}
The closure is the SV that closes the SV sequence, returning everything to normal. This is so that the rest of the chart isn't affected by \textbf{Act or Counteract SV}.

\subsection{Breakdown of Effects}

\subfile{../part_4/figs/se_fig1.tex}

This is one of the most common and simpler effects to show. This produces a "jump" effect, where it speeds up, then comes to stop. The "jump" is mainly shown by \textbf{1.5x}, hence it's the Act. The "fix" is reliant on the value of your Act, it is calculated to be \textbf{0.5x}.

Last but not least, the closure, the scroll speed multiplier that you want the rest of the chart to go at. \newline
Here's two more examples on how this works

\subfile{../part_4/figs/se_fig2.tex}
\subfile{../part_4/figs/se_fig3.tex}

\subsection{The Half-Half Rule}

This rule works \textbf{if and only if} \textit{your second SV is halfway in between the first and third}.

Notice how all of the effects have certain values that work, here's short list of valid SVs that works in those situations, see what you can deduce from this

\subfile{../part_4/figs/se_table1.tex}

We can say that...

\[ \frac{SV_{Act} + SV_{C.Act}}{2} = SV_{Closure} = SV_{Average} \]

OR

If your $SV_{Average} = 1.0x$, then...

\[ SV_{Act} + SV_{C.Act} = 2.0 \]

This is the \textbf{Half-Half Rule}, all SVs created with this \textbf{will average to 1.0}. 
\newpage

\subsubsection{How it Works}
Avoiding the more complicated mathematics, we can explain this with some diagrams.

\subfile{../part_4/figs/se_calcfig1.tex}

This is a simplified diagram for a \textbf{Half-Half Rule} situation, all SVs are represented in the boxes. When we \textbf{displace} 0.5x SV we can easily shift it to the other SV so it still has \textbf{the same area}. In other words, if it has \textbf{the same area}, it will always be \textbf{Even Visual Spacing}.

Let's take a lot at a few more examples:

\subfile{../part_4/figs/se_calcfig2.tex}

This is simple mathematics, we actually don't need diagrams. However, this is essential in helping you understand the more dynamic cases, where you can't use the \textbf{Half-Half Rule}.