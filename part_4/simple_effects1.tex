If you've read the previous topic, you should be able to understand this topic better, if you haven't, that's fine, I'll still guide you through from square one.

\noindent
Before we go into putting multiple SVs together, we have to understand the effects of SVs, down to the core. We'll start with something that is not commonly talked about: \textbf{Visual Spacing}

\section{Visual Spacing}
Visual Spacing is the space in between notes that players will see, this gives the indication on if the note is a certain snap. Eg. $1/4$ notes will always have \textbf{half} the spacing of $1/2$ since it's \textbf{twice as fast}.\newline
Consider this scenario. Visually, all of these diagrams look like they are 1/4 snaps

\subfile{../part_4/figs/se_vsfig1.tex}

It's obvious how that works, but watch what happens when you \textbf{add SVs and move notes around} on the next scenario. \newpage

\subfile{../part_4/figs/se_vsfig2.tex}

\subsection{Uneven Visual Spacing}

With regards to what an SV does, a 0.5x SV will make \textbf{1/2 spacing} visually look like \textbf{1/4 spacing} but it will STILL play as if it was on \textbf{1/2 spacing}. This is an \textit{uneven} SV Effect because it distorts the \textbf{Visual Spacing}. 

\subsection{Downsides of Uneven Visual Spacing}

With \textbf{unexpected} uneven visual spacing/SV, players will be \textbf{expecting a certain snap}, but instead, is \textbf{greeted with another snap}.\newline
Referring to the previous diagram, we can see that the player will be \textbf{expecting a 1/4 snap} for all 4 notes. However, since a \textbf{0.5x} SV was implemented, \textbf{the first 2 notes will play as if it was 1/2 apart}
\noindent
Here's another example for easier digestion

\subfile{../part_4/figs/se_vsfig3.tex}

Now you understand the effects of SV on notes, we can go onto how to make SVs that create \textbf{Even Visual Spacing}, in the following topic.\newpage

